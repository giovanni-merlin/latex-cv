%%%%%%%%%%%%%%%%%%%%%%%%%%%%%%%%%%%%%%%%%
% Developer CV
% LaTeX Class
% Version 2.0 (12/10/23)
%
% This class originates from:
% http://www.LaTeXTemplates.com
%
% Authors:
% Omar Roldan
% Based on a template by  Jan Vorisek (jan@vorisek.me)
% Based on a template by Jan Küster (info@jankuester.com)
% Modified for LaTeX Templates by Vel (vel@LaTeXTemplates.com)
%
% License:
% The MIT License (see included LICENSE file)
%
%%%%%%%%%%%%%%%%%%%%%%%%%%%%%%%%%%%%%%%%%

%----------------------------------------------------------------------------------------
%	PACKAGES AND OTHER DOCUMENT CONFIGURATIONS
%----------------------------------------------------------------------------------------

\documentclass[9pt]{developercv} % Default font size, values from 8-12pt are recommended
\usepackage{multicol}
\setlength{\columnsep}{0mm}
%----------------------------------------------------------------------------------------
\usepackage{lipsum}  


\begin{document}

%----------------------------------------------------------------------------------------
%	TITLE AND CONTACT INFORMATION
%----------------------------------------------------------------------------------------

\begin{minipage}[t]{0.5\textwidth} 
	\vspace{-\baselineskip} % Required for vertically aligning minipages
	
	{ \fontsize{16}{20} \textcolor{black}{\textbf{\MakeUppercase{Giovanni Merlin}}}} % First name
	
	\vspace{6pt}
	
	{\Large MSc Physics graduate} % Career or current job title
\end{minipage}
\hfill
\begin{minipage}[t]{0.2\textwidth} % 20% of the page width for the first row of icons
	\vspace{-\baselineskip} % Required for vertically aligning minipages
	
	% The first parameter is the FontAwesome icon name, the second is the box size and the third is the text
	%\icon{Globe}{11}{\href{http://www.google.com}{portafolio.com}}\\ 
    \icon{Linkedin}{11}{\href{https://www.linkedin.com/in/giovanni-merlin-a4461a236/}{in/giovanni-merlin}}\\
    \icon{MapMarker}{11}{Vicenza, Italy}\\
	
\end{minipage}
\begin{minipage}[t]{0.27\textwidth} % 27% of the page width for the second row of icons
	\vspace{-\baselineskip} % Required for vertically aligning minipages
	
	\icon{Envelope}{11}{\href{mailto:gmerlin199@gmail.com}{gmerlin199@gmail.com}}\\	
    \icon{Github}{11}{\href{https://github.com/giovanni-merlin}{github.com/giovanni-merlin}}\\
    %\icon{LinkedinSquare}{11}{\href{https://www.linkedin.com}{/in/your-personal-url}}\\    
    
\end{minipage}

\hfill

%----------------------------------------------------------------------------------------
%	INTRODUCTION, SKILLS AND TECHNOLOGIES
%----------------------------------------------------------------------------------------

\begin{minipage}[t]{0.46\textwidth}
    \cvsect{Summary}
	\vspace{-6pt}
 
    %Dummy text
	Motivated M. Sc. Physics of Data graduate (25) with a strong foundation in Physics, Mathematics, and Statistics. Passionate about solving complex physics problems with the aid of programming and data analysis. Enthusiastic about hands-on laboratory work and driven by a curiosity to develop innovative solutions to challenging scientific and technical problems. \\
 
\end{minipage}
\hfill % Whitespace between
\begin{minipage}[t]{0.465\textwidth}
    \cvsect{Skills}
    \vspace{-6pt}
    
    \begin{minipage}[t]{0.2\textwidth}
        \textbf{Languages:}
    \end{minipage}
    \hfill
    \begin{minipage}[t]{0.73\textwidth}
      Python, C, R, Mathematica, \LaTeX, Bash.
    \end{minipage}
    \vspace{4mm}
    
    \begin{minipage}[t]{0.2\textwidth}
        \textbf{Technologies:}
    \end{minipage}
    \hfill
    \begin{minipage}[t]{0.73\textwidth}
      Git, Anaconda, PyTorch, Numpy.
    \end{minipage}
    
\end{minipage}

%----------------------------------------------------------------------------------------

%-----------------------------------------------------------------
%	EXPERIENCE
%----------------------------------------------------------------------------------------
\vspace{-10 pt}
\cvsect{Experience}
\begin{entrylist}
	\entry
        {07/2024 -- 12/2024}
		{Research Intern}
		{University of Strasbourg (FR)}
            {\textit{Exotic Quantum Matter Group - Prof. Shannon Whitlock, European Center of Quantum Sciences (CESQ)}

            - Upgraded the experimental control system integrating state-of-the-art arbitrary waveform generators, enabling precise manipulation of neutral atoms in optical tweezers.

            - Developed a real-time optimization tool for fine-tuning experimental parameters.
            }	
\end{entrylist}

\begin{entrylist}
	\entry
        {04/2022 - 08/2022}
		{ Research Intern}
		{University of Padua}
            {\textit{Quantum Future Group - Dr. Francesco Vedovato, Department of Information Engineering}

            - Developed a fine-tracking system for the local telescope, imporoving the efficiency of free-space satellite Quantum Key Distribution.
            }
\end{entrylist}

%----------------------------------------------------------------------------------------
%	EDUCATION
%----------------------------------------------------------------------------------------
\vspace{-10 pt}
\cvsect{Education}
\begin{entrylist}
    \entry
		{10/2022 - 04/2025}
		{Master's Degree in Physics of Data, \small{mark 110/110 with honours} }
		{University of Padua}
		{Merges and innovates the educational offers from Physics and Data Science.}
    \entry
		{10/2018 - 10/2022}
		{Bachelor's Degree in Physics, \small{mark 106/110}}
		{University of Padua}
		{Provides solid foundation in physics, mathematics, and statistics.}

\end{entrylist}

%----------------------------------------------------------------------------------------

%	Projects
%----------------------------------------------------------------------------------------
\cvsect{Projects}
\begin{entrylist}
    \entry
        {}
        {\href{https://github.com/giovanni-merlin/NNDL-HAR-project}{Human Activity Recognition and Person Identification via Contrastive Learning strategies}}
        {PyTorch}
		{Applied Deep Learning techniques for the classification of spectrograms generated from Wi-Fi router signals. Trained a ResNet-18 model to accurately identify both the activity performed and the individual, even when performed in an unfamiliar environment.}
    \entry
        {}
        {\href{https://github.com/diegobonato/approximate-bayes-sir}{Approximate Bayes Computation: an application to the SIR model}}
        {Python}
		{Applied a likelihood-free inference technique (ABC) to infer the hidden parameters of the SIR model by simulating synthetic data.}
    \entry
        {}
        {Time-scale correction of radio-nuclide data}
        {Python, R}
		{Corrected a Be-10 timescale employing bayesian inference by means of Monte Carlo sampling techniques (Hamiltonian Monte Carlo and Gibbs sampling).}


\end{entrylist}

%-----------------------

%	LANGUAGES
%----------------------------------------------------------------------------------------
\vspace{-10 pt}
	\cvsect{Languages}
    \vspace{-6pt}
    
    \hspace{26mm} \textbf{English} - C1 (Cambridge) \; \textbf{ Italian} - native

%----------------------------------------------------------------------------------------
%----------------------------------------------------------------------------------------
%	Extra
%----------------------------------------------------------------------------------------
\vspace{-10 pt}
\cvsect{Extra}
\begin{entrylist}

	\entry
	{2019 - present}
	{Tutor}
	{}
	{Tutor for final examinations and make-up exams.}

\end{entrylist}

\end{document}
